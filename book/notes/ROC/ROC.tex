\documentclass[12pt]{article}
%\documentclass[preprint]{elsarticle}
%%% Following line makes it easier for me to do editing on a ``big screen''
%\addtolength{\voffset}{-1.5in}
\voffset 0.0cm
\topmargin 0.0cm
\headheight 0.0cm
\headsep 0.0cm
\hoffset 0.0cm
\textheight 9in
\textwidth 6.5in
\oddsidemargin 0.0cm

%\usepackage{savetrees}
\usepackage[dvips]{graphicx}
\usepackage{verbatim}
\usepackage[sumlimits,]{amsmath}
\usepackage{amsfonts}
\usepackage{amsthm}
\usepackage{amssymb}
\usepackage{mathrsfs}
\usepackage{srcltx}
\usepackage{bbm}
\usepackage{algorithm,algorithmic}
\usepackage{hyperref}
\usepackage{url}
\usepackage{authblk}

\usepackage{graphicx,epsfig}
%\usepackage{subfigure}
%\usepackage{amstext,amsmath,amssymb,amsfonts,amsthm,datetime}
%\usepackage{times}
%\usepackage{natbib}
%\def\newblock{\hskip .11em plus .33em minus .07em}  %Use this with natbib to avoid a ``newbloc undefined'' error
%\setlength\textwidth{37.2pc}
%\setlength\textheight{58pc}
%\setlength\topmargin{-30pt}
%\addtolength\oddsidemargin{-1.2cm}
%\addtolength\evensidemargin{-1.2cm}


%%%Boldface and special characters%%%%%
%%roman, boldface, cal
\def \bfd { {\mathbf{d}}}
\def \bfD { {\mathbf{D}}}
\def \bft { {\mathbf{t}}}
\def \bfx { {\mathbf{x}}}
\def \bfX { {\mathbf{X}}}
\def \calM { {\mathcal M}}
\def \calA {\mathcal{A}}
\def \calD {\mathcal{D}}
\def \calK {\mathcal{K}}
\def \calL {\mathcal{L}}
\def \calN { {\mathcal N}}
\def \calQ {\mathcal{Q}}
\def \calR {\mathcal{R}}
\def \calU { {\mathcal{U}}}
\def \d { {\,\mbox{d}}}
\def \dl { {\,\mbox{dl}}}
\def \dP { {\,\rm{d}\rmP}}
\def \dQ { {\,\mbox{dQ}}}
\def \Pbar { {\overline{\rm{P}}}}
\def \Q { {\mbox{Q}}}
\def \dm { \, \mbox{d}m}
\def \dq { \, \mbox{d}q}
\def \ds { \,\mbox{d}s}
\def \dt { \,\mbox{d}t}
\def \du { \,\mbox{d}u}
\def \dv { \,\mbox{d}v}
\def \dw { \, \mbox{d}w}
\def \dx { \, \mbox{d}x}
\def \dy { \, \mbox{d}y}
\def \dz { \,\mbox{d}z}
\def \iid { {\mbox{i.i.d.}\,}}
\def \rmP {\mathrm{P}}
\def \supp { {\mbox{supp}}}
\def \trace { {\mbox{Trace}}}
\def \st {:\,}


%%% common math-mode items %%%
\def \as { {\emph{a.s.}}}
\def \E { {\mathbb{E}}}
\def \eps { {\varepsilon}}
\def \indep { {\perp\!\!\!\perp}}
\def \logit { {\mbox{logit}}}
\def \one { \mathbf{1}}
\def \what { {\hat w}}
\def \xbar { {\bar x}}
\def \xhat { {\hat x}}


%%Operators operators commands
\newcommand{\HRule}{\noindent\rule{\linewidth}{0.5mm}}
\newcommand{\Cov}[2]{ \mbox{Cov}\left( #1,#2 \right) } 
\newcommand{\CovP}[2]{ \mbox{Cov}_\rmP\left( #1,#2 \right) } 
\newcommand{\Exp}[1]{ \E\left\{ #1 \right\} } 
\newcommand{\ExpQ}[1]{ \E_Q\left\{ #1 \right\} } 
\newcommand{\ExpP}[1]{ \E_\mathrm{P}\left\{ #1 \right\} } 
\newcommand{\ip}[2]{ \langle #1,\,#2 \rangle}   %Mapping is ` i 
\newcommand{\KL}[2]{ \mbox{KL}[#1 \,||\, #2]}   %Mapping is 
\newcommand{\mean}[1]{ \langle #1 \rangle}      %Mapping is ` m
\newcommand{\Var}[1]{ \mbox{Var}\left\{ #1 \right\} } 
\newcommand{\VarP}[1]{ \mbox{Var}_\mathrm{P}\left\{ #1 \right\} } 
\def \dint { \displaystyle\int}
\def \g { {\,|\,}}
\def \p {\partial}


%%%%spaces such as R^n %%%
\def \Nat {{\mathbb N}}
\def \Rone { {{\mathbb R}}}
\def \Rd { {{\mathbb R}^d}}
\def \Rm {{\mathbb R}^m}
\def \Rn { {{\mathbb R}^n}}
\def \RN {{\mathbb R}^N}

%%%%Theorem Environments%%%%%%%%%%%%%%
\newtheorem{theorem}{Theorem}[section]
\newtheorem*{theorem*}{Theorem}
\newtheorem{proposition}{Proposition}[section]
\newtheorem{lemma}{Lemma}[section]
\newtheorem{corollary}[theorem]{Corollary}

\theoremstyle{definition}
\newtheorem*{def*}{Definition}
\newtheorem{definition}{Definition}[section]
\newtheorem{conditions}{Conditions}[section]
\newtheorem{assumptions}{Assumptions}[section]

\theoremstyle{remark}
\newtheorem*{remark*}{Remark}
\newtheorem{remark}{Remark}[section]
\newtheorem{claim}{Claim}
\newtheorem*{claim*}{Claim}


\begin{document}
\title{ROC Curves}
%\author{Guillaume Bal\thanks{Department of Applied Physics and Applied Mathematics, Columbia University, 200 S.W. Mudd building, 500 W. 120th street, New York NY, 10027; 212-854-4731, gb2030@columbia.edu}, Ian Langmore \thanks{Corresponding author.  Department of Applied and Applied Mathematics, Columbia University, 200 S.W. Mudd building, 500 W. 120th street, New York NY, 10027; 415-272-6321, ianlangmore@gmail.com }, Youssef Marzouk\thanks{Dept. of Aeronautics and Astronautics, MIT}}
%\author{
% Ian Langmore,\thanks{Opera Solutions
%    ianlangmore@gmail.com } 
%    }
\author[1]{Ian Langmore\thanks{ianlangmore@gmail.com}}
%\author[1]{Ming Gu\thanks{mgu@operasolutions.com}}
%\affil[1]{Global Market Analytics, Opera Solutions}
\maketitle

\tableofcontents
\abstract{}


\medskip


\section{Notation}
Following notation in \cite{FawROC}, we define $Y,N$ as the sets of outcomes that are predicted positive/negative, and $p, n$ as the outcomes that are positive/negative.


\section{ROC for $-x$}
Consider a logistic regression model that predicts the probability of $p$, as a function of $x$ as $\logit(x)$.  

For every threshold level, the models with $x$ and $-x$ give opposite predictions.
\begin{align*}
  \rmP_x[Y\g p] &= 1 - \rmP_x[N\g p] = 1 - \rmP_{-x}[Y\g p]\\
  \rmP_x[Y\g n] &= 1 - \rmP_x[N\g n] = 1 - \rmP_{-x}[Y\g n].
\end{align*}
The ROC curve of the $x$ model is the graph $(\rmP_{x}[Y\g p], \rmP_{x}[Y\g n]) = (w, f(w))$.  The ROC curve of the $-x$ model is the graph $(z, g(z))$.  The above equations show that
\begin{align*}
 w &= 1-z, \qquad f(w) = 1 - g(z) = 1 - g(1-w), 
\end{align*}
therefore
\begin{align*}
  AUC_x &= \int_0^1 f(w)\dw = \int_0^1 \left( 1 - g(1-w) \right)\dw = 1 - \int_0^1 g(1-w)\dw = 1 - \int_0^1 g(z)\dz\\
  &= 1 - AUC_{-x}.
\end{align*}
 
\bibliographystyle{plain}
\bibliography{../../../bibliography}

\end{document}
